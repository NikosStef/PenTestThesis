\chapter{Introduction}

Since the infancy of computers, hackers have been creatively solving problems. In the late 1950s, the MIT model railroad club was given a donation of parts, mostly old telephone equipment \cite{erickson2008}. The club's members used this equipment to rig up a complex system that allowed multiple operators to control different parts of the track by dialing in to the appropriate sections. They called this new and inventive use of telephone equipment hacking; many people consider this group to be the original hackers.

Today, the term has quite a different meaning. When people think of computer hackers, they think of computer experts who are adept at reverse engineering computer systems. They might think of malicious hackers who aspire to break into networks to destroy or steal data, or of ethical hackers who are hired to test the security of a network. Often, these ethical hackers, or penetration testers, mimic the same techniques as a malicious hacker since the best way to stop a criminal is to think the way a criminal thinks. It is not enough to install burglar alarms and fences and assume that you are safe from burglary; to effectively stop a burglar, you must predict the actions a burglar would take. Likewise, to prevent against malicious hackers, you must think like a malicious hacker. One of the best ways that companies are assessing their security against attacks is by hiring outside security firms to attempt to penetrate their networks. This is why the need for penetration testing is immense but also very simple \cite{whitaker2006}.


\section{Motivation}

Risk management is the ongoing process of identifying, assessing, and responding to risk. To manage risk, organizations should understand the likelihood that an event will occur and the potential resulting impacts. With this information, organizations can determine the acceptable level of risk for achieving their organizational objectives and can express this as their risk tolerance. Nowadays, enterprises find it difficult to protect the confidential information of clients while maintaining a public Internet presence. To mitigate risks, it is customary for companies to turn to penetration testing for vulnerability assessment. Penetration testing is the practice of a trusted third-party company attempting to compromise the computer network of an organization for the purpose of assessing its security. By emulating a live attack, executives can witness the potential of a malicious attacker gaining entry or causing harm to the data assets of that company.

Today, news of security threats or security breaches dominate headlines on a daily basis. Over the past years, we have been hearing that hacking attacks and website defacement are becoming more frequent and are happening to thousands of companies worldwide. Some of the few high profile organizations who were victims of massive network security breaches in 2018 are:
\begin{center}
\begin{itemize}

\item \textbf{Adidas}:  On June 26, Adidas became aware that an unauthorized party claims to have acquired limited data associated with certain Adidas consumers. More information is available on the official Adidas website \footnote{Adidas: \url{https://www.adidas-group.com/en/media/news-archive/press-releases/2018/adidas-alerts-certain-consumers-potential-data-security-incident/}}. 

\item \textbf{Under Armour}: About 150 million users of its nutrition-tracking app, MyFitnessPal, had been hacked. The stolen data includes account user names, email addresses and scrambled passwords for the popular MyFitnessPal mobile app and website, Under Armour said in a statement \footnote{Under Armour: \url{http://fortune.com/2018/03/29/myfitnesspal-password-under-armour-data-breach/}}.

\item \textbf{FedEx}: Thousands of FedEx customer records exposed by unsecured server. More details can be found online \footnote{FedEx: \url{https://www.usnews.com/news/world/articles/2018-02-15/thousands-of-fedex-customer-records-exposed-by-unsecured-server}}.

\item \textbf{T-Mobile}: T-Mobile cyber security staff detected the attack a short time after it began. In a statement to Motherboard, a T-Mobile spokesperson said that "less than 3\%" of the company's roughly 76 million subscribers was accessed \footnote{T-Mobile: \url{https://www.forbes.com/sites/leemathews/2018/08/24/t-mobile-hackers-swipe-data-on-2-million-subscribers/\#5cf3f8a7a523}}.

\item \textbf{British Airways}: It is thought the number of payments compromised could be up to 400,000 and British Airways confirmed hackers had obtained names, addresses, credit card numbers, expiry dates and the three-digit security codes on the backs of cards - plenty to make a fraudulent payment \footnote{British Airways: \url{https://www.telegraph.co.uk/news/2018/09/07/british-airways-hacking-customers-cancel-credit-cards-airline/}{The Telegraph}}.

\item \textbf{Facebook}: The breach was the largest in the company's 14-year history. The attackers exploited a feature in Facebook's code to gain access to user accounts and potentially take control of them \footnote{Facebook: \url{https://www.nytimes.com/2018/09/28/technology/facebook-hack-data-breach.html}}.

\item \textbf{United States Air Force}: A hacker penetrated an Air Force captain's computer to steal sensitive information about US military drones \footnote{US Air Force: \url{https://edition.cnn.com/2018/07/10/politics/us-reaper-drone-materials-hacker-theft/index.html}}.

The incidents mentioned above are just a small subset of the reported events.

\end{itemize}
\end{center}
The statistics on threats posed by hackers are sobering. A recent report by the RAND Corporation\footnote{The RAND Corporation is a nonprofit institution that helps improve policy and decision-making through research and analysis. For further information, see \url{https://www.rand.org} .} suggests that in one year as many as 65 million people in the USA alone have had their personal data breached in some way or other, and that cyber-crime generates billions of dollars in revenue each year \cite{rand2016}.

With cyber attacks becoming the norm and the advancement of technology and computers, the necessity to undertake regular vulnerability scans to identify vulnerabilities has increased and ensuring on a regular basis that the cyber controls are working is a priority to organizations and their executives. No matter how much patching an administrator or an engineer does to the environment; the systems can still be vulnerable to attack. This is where Penetration Testing comes in.

\section{Research Topics}
The scope of this thesis is to investigate and question some problem statements in the field of Penetration Testing. We examine important quality aspects of Pen-Testing like commonly used tools, community's best practices, architecture of a test and evaluation of its performance as well as reporting. The technology used consists of mostly open-source tools.
In this thesis, we explore the following research topics:
\begin{enumerate}
\item Identify the different types of Penetration Test and Document Testing Methodologies.
\item Investigate and Compare Penetration Testing tools and techniques.
\item Design and Perform Penetration Testing attacks on an Isolated Network Laboratory. 
\item Understand the importance of analysis and reporting and the necessity of those to properly translate technical findings into risk mitigation strategies that will improve security posture.
\end{enumerate}

\section{Outline}

Chapter 2 introduces the act of Penetration Testing and its origin, necessary information for understanding the fundamentals of Penetration Testing. We then proceed to explain the different types of testing and how those will benefit the general security posture of the system, along with the different methodologies and rules a tester may implement.

In Chapter 3, we investigate and present the different types of software a tester can use along with their role in the general process of testing. A comparison between the tools is presented in order to analyze and research the different features they offer.

In Chapter 4, Network-based attacks are emulated in a virtual environment. Every attack is described and then executed, with the goal of this chapter being the representation of the potential of individual's code and how they can be extended to become very powerful.

In Chapter 5, we focus on the necessity of a thorough report, where the importance of proper risk evaluation is presented to the reader. Also, a general format of reporting and key characteristics of a complete report are presented and analyzed during this chapter.

Lastly, in Chapter 6, we reiterate and summarize on the entire perspective on Penetration testing, highlight future work that can be done to further improve this thesis in different ways.
