\chapter{Analysis and Reporting}

\section{Findings and Evidence}
Proper documentation and evidence is an important part of a well-rounded penetration test. Collecting and saving evidence does not have a standard report template for penetration testers to follow. Examples of record keeping are log files, notes, description of events and impact, screenshots and many more depending on what is more comfortable to the tester but most importantly the, suggested by the organization, reporting model which the tester should always be in accordance with \cite{kim2018}. A very effective reporting model and environment would be a web server, where the penetration test team/Red Team will be able to record, track, analyze and review the whole team's steps and findings, with the potential of making the final report more thorough and easier to assemble. Without proper evidence and information, the audience of the final report cannot be sure about the findings and their severity, having a limited view of the test's outcome and making unfit decisions as a consequence. Furthermore, other testers will not be able to contribute to the work or even take over when needed.


\section{Risk Evaluation}
Discovering vulnerabilities is important, but being able to estimate the associated risk is just as important. Security concerns might be identified during a penetration test. On the other hand, problems may not be discovered until the system or application is actually compromised and/or under attack. Members of an organization's Red Team should be in a position to understand the potential impact of a risk as much as possible and prioritize accordingly. Without proper risk evaluation, an opportunity might rise for the attacker. If a vulnerability or an incident is not properly assessed or mishandled, the attacker will have the possibility to penetrate the system, having a harmful impact as a result \cite{owasp2014}\cite{nist2008}.

A risk analysis model is based on the factors that it can be broken down to. Therefore, a risk model can be something objective, where every penetration tester or security analyst can follow his own, depending on his/her experience and his/her understanding of risk and vulnerabilities. 
The, presented below, model follows the structure and approach of the OWASP Testing Guide\footnote{See: \url{https://www.owasp.org/index.php/OWASP_Testing_Project}} and associates risk with the \textit{impact} of a successful exploit and the \textit{likelihood} of the attack to take place \cite{owasp2014}. Those are the two fundamental elements that one should take into consideration, in order to estimate the potential risk.

\subsubsection{Estimating Likelihood}

After identifying a risk, the first step to determining its severity, is to estimate its likelihood of exploitation against that risk. The preciseness of the estimation is not the goal in this step, but the general understanding of whether the likelihood is low, medium, or high is sufficient. The factors that will help the estimation can be split into two sets, where the first set is related to the threat agent involved. At this point, it is important to note that there might be multiple threat agents able and willing to execute the attack, so it is usually best to use the worst-case scenario \cite{owasp2014}.

The factors related to the threat agent are the following:
\begin{itemize}
    \item The technical skill level of the attacker.
    \item Motive behind the attack and potential reward.
    \item Resources and opportunities required (i.e. computer and software resources, any kind of access).
    \item How large is the group of attackers? (e.g. anonymous internet users to system administrators).
\end{itemize}

Of course, every vulnerability is different, such is every risk. Thus, the second set of factors is related to the vulnerabilities involved and their different characteristics. The tester's goal is to determine the difficulty of discovering and abusing a certain weakness in the system. The associated factors are \cite{owasp2014}:

\begin{itemize}
    \item Ease of discovery.
    \item Ease of exploit.
    \item How well-known is this vulnerability to the group of threat agents?
    \item Intrusion Detection and the probability of the exploit to  be detected.
\end{itemize}

\subsubsection{Estimating Impact}

When considering the impact of a successful attack, two aspect to be considered. The first is the technical part, which is the impact on the application, the data it uses, and the functions it provides. The other aspect is the impact on the business and company operating the application. Ultimately, the business impact is more important. Although, since access to information might be restricted, the business consequences cannot be estimated completely but  providing as much detail about the technical risk will enable the appropriate business representative to make a decision about the business risk \cite{owasp2014}.

Technical impact factors resemble the traditional security areas of concern, which are confidentiality, integrity, availability and accountability. Those factors guide the tester to a better Understanding the degree of the impact, if the system were to be attacked \cite{owasp2014}.

\begin{itemize}
    \item Loss of confidentiality and sensitivity of data disclosed.
    \item Loss of integrity and potential damage to data and information.
    \item Loss of availability and interruption of service.
    \item Loss of accountability and traceability
\end{itemize}

Business impact is what supports the identified risks and shows the potential damage to the audience of any level. Accordingly, the business risk justifies investment in fixing security problems and protecting what is important to the organization. The business impact derives from the technical impact and damage on the second can have an immediate effect on the other. The following factors are crucial and common for many businesses \cite{owasp2014}.

\begin{itemize}
    \item Financial damage and effect on profit.
    \item Reputation damage.
    \item Non-compliance and violation.
    \item Privacy violation and personal information theft.
\end{itemize}

\subsubsection{Determining the Severity}

To make a final result, adjusting the model to the organization's priorities is necessary. The tester can choose to add factors that may be more significant. Also, some factors have different importance to the company that other, therefore, weighting factors will help to emphasize the crucial ones. These steps will produce better results, tailored to the organization's priorities, and excess investments and actions might be avoided \cite{owasp2014}.

\section{Reporting}
Upon completion of analysis, a report should be generated that identifies system, network, and organizational vulnerabilities and their recommended mitigation actions. For a penetration test report to be successful, the risks and their impact must be portrayed in such a way so that the listener can properly interpret the severity of the situation and as a result take more rounded and confident decisions \cite{nist2008}\cite{pci2015}.

As proposed by the Security Standards Council\footnote{Visit: \url{https://www.pcisecuritystandards.org/documents/Penetration-Testing-Guidance-v1_1.pdf}}, a penetration test report should abide with the following format \cite{pci2015}\cite{weidman2014}:

\begin{itemize}
    \item \textbf{Executive Summary.} Brief summary of the penetration test scope and results, business impact and high-level strategic roadmap to rectify the risks. Target audience is mainly non-technical executives.
    \item \textbf{Scope.} Detailed definition of the scope, techniques and targets.
    \item \textbf{Methodology.} The methodology and procedure implemented to execute the penetration test. Tools used and actions performed during the test.
    \item \textbf{Limitations.} Restrictions imposed during the test, e.g bandwidth restrictions, testing hours, special legal requirements, etc.
    \item \textbf{Test Presentation.} Complete documentation of the test with the according pictures and evidence provided.
    \item \textbf{Test Results.} Summarization of the test and results.
    \item \textbf{Findings.} A listing of all identified vulnerabilities accompanied with their ease of exploitation and risk rating. Also, contains positive test findings.
    \item \textbf{Mitigation.} Risk mitigation and security enhancing suggestions. Examples of mitigation actions include policy, process, and procedure modifications; security architecture changes; deployment of new security technologies, OS and application patches, and many more depending on the vulnerabilities found and more important, the business' goals, plans and policy.
\end{itemize}