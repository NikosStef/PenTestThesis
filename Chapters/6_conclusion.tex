\chapter{Conclusion and Future Work}

\section{Conclusion}

A successful penetration test has a strong and undisputed connection with the underlying methodology. Different methodologies can produce different results, which makes the definition of the methodology an important part of the test. To properly lay out the performance of the test, a respectively thorough report and presentation is necessary. One of the goals of this thesis was to portray and explore the different types of proposed methodologies and tools. Tools can be divided into categories, depending on the type of vulnerabilities sought after and the technologies that they are found at. As mentioned before, every tool may expose a distinct variety of security flaws, so it is recommended to use a wide spectrum of tools and scripts during testing. However, the final result and the general success of the test is mostly dependent to the penetration tester. Even with the most powerful tools, a test may not be adequate to its full potential if the user of those tools is not able to know how to use them in his/her favor. Tools and techniques can just be a matter of choice and expertise.

Another goal of this thesis was to introduce the readers to the potential of scripting and coding. Tools can sometimes be inadequate or decorative. Coding can be an ace up the tester's sleeve, since it can either fill in the gaps from the tools or even carry out the whole test. The effectiveness is dependent on the tester's skills and experience. Throughout chapter 4, \textit{Python 3}\footnote{Visit the official organization website at \url{https://www.python.org}} was the coding language in which the scripts were written at, and \textit{Scapy}\footnote{Read more at \url{https://pypi.org/project/scapy/}} was the main library used for the purposes of the labs. No language or library is better than another because of the various uses each one of them has. It is on the user's hands to display their potential.

In conclusion, tools and methodologies, if properly utilized, can prove their usefulness for discovering the potential weaknesses of the system and the way to exploit them. Penetration testing plays its own part in building a strong security posture and neither is an alternative to other security measures nor a guarantee for system and data protection. In today's world, the number of information is rising with a vast and increasing rate. For that reason, information security is and will be relevant. Threats and vulnerabilities are evolving with new technologies but also attackers are constantly finding new ways to approach and perform cyber attacks. One should also keep changing and evolving, constantly learning and exploring along with technological advancement.

\section{Future Work}

This work can be extended in different directions:

\begin{itemize}
    \item Automation of Penetration Testing and a complete security testing solution can be an extension of this thesis work. As mentioned above, automation has its benefits, including time and less human resources. This addition can  empower the tester's strength by giving him time and possibility to focus on complementing the testing solution or even amplify the test and the solution itself.
    \item A very important element in Penetration Testing is the human factor. This thesis can be extended to consider this major risk point of the general security posture of an organization. Efforts can be done by integrating social engineering tools and techniques and the importance of the human behind the computer.
    \item New technologies arise as time passes by and Internet of Things are getting more and more popular and advanced. Their security, though, is a very controversial topic, since many attacks seem to have started through those. How those can be examined during a test and secured can be another subject to expand this thesis.
    \item As Cloud Computing\footnote{Cloud computing and storage provides users with capabilities to store and process their data in third-party data centers.} advances, security issues of it fall into two broad categories, the provider's and the customer's. The bridge in between those are Containers (e.g. Docker\footnote{Visit \url{https://www.docker.com} for further information.}, Kubernetes\footnote{Learn more about Kubernetes at \url{https://kubernetes.io/docs/concepts/overview/what-is-kubernetes/}} etc.). The container pipelines, deployment environment and many more, can be studied, examined and implemented to the extension of this thesis work, as well as cloud computing in general. 
\end{itemize}

